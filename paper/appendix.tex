%%%%%%%%%%%%%%%%%%%%%%%%%%%%%%%%%%%%%%%%%%%%%%%%%%%%
%    Canadian AI Latex Template    %
%%%%%%%%%%%%%%%%%%%%%%%%%%%%%%%%%%%%%%%%%%%%%%%%%%%%
\documentclass[10pt]{cai}
\captionsetup{font=small}
\begin{document}
% Editorial staff will replace the following values:
% 1. Conference Year
% 2. Issue number
% 3. Article DOI
\def\conferenceyear{2025}
\volumeheader{38}{0}%{00.000}
\begin{center}

\title{Eco Tracker Appendix}
\maketitle

\thispagestyle{empty}

% Add Authors and Affiliations in the camera ready
% for the double blind review, please leave this section as is 
\begin{tabular}{cc}
First Author\upstairs{\affilone,*}, Second Author\upstairs{\affilone}, Third Author\upstairs{\affilthree}
\\[0.25ex]
{\small \upstairs{\affilone} Affiliation One} \\
{\small \upstairs{\affiltwo} Affiliation Two} \\
{\small \upstairs{\affilthree} Affiliation Three} \\
\end{tabular}
  
% Replace with corresponding author email address
\emails{
  \upstairs{*}corresponding\_author@example.ca 
}
\vspace*{0.2in}
\end{center}

\begin{abstract}
This is an appendix of additional material that we don't have room to put into the paper.
This material will allow us to expand if needed and is here to test our own understanding of the the underlying mechanisms.

\end{abstract}

% add your keywords
\begin{keywords}{Keywords:}
Internet of Things (IoT), Reinforcement Learning, Cattle Monitoring.
\end{keywords}
\copyrightnotice

\section{Simple Battery Drain Baseline}
An initial experiment to verify that an agent is capable of learning involved a simple battery drain test.
The battery was fully charged and the agent was run for as long as needed to discharge the battery while trying to maximize message collection/transmission.
The agent was compared to a simple baseline static stragy of alternating transmission and collection.
The baseline strategies are as follows:
\begin{itemize}
  \item Baseline 0: Take the Transmission action every time
  \item Baseline 1: Collect once, transmit once
  \item Baseline 2: Collect twice, transmit once
  \item Baseline 3: Collect three times, transmit once
  \item Baseline 4: Collect four times, transmit once
\end{itemize}

\begin{table}[h]
  \centering
  \caption{Comparison of Baseline Models and RL Agent}
  \begin{tabular}{lcccccc}
      \toprule
      & \textbf{Baseline 0} & \textbf{Baseline 1} & \textbf{Baseline 2} & \textbf{Baseline 3} & \textbf{Baseline 4} & \textbf{Agent} \\
      \midrule
      Collection + Transmission Action & 205 & 117 & 81  & 62  & 50  & 54  \\
      Collection Action                 & 0   & 117 & 162 & 186 & 200 & 198 \\
      Total Messages                     & 205 & 234 & 243 & 248 & 250 & 252 \\
      \bottomrule
  \end{tabular}
  \label{tab:batter_drain_result}
\end{table}

The results are shown in Table \ref{tab:batter_drain_result}.
The baseline is able to send the most messages if it collects four times and sends three times.
This would seem like the most optimum strategy, but the static schedule gets into trouble when the battery runs out and the entire queue of messages is lost.
The agent learns an optimal strategy of collecting four times and sending once except when the battery is low.
In this case, the agent transmits more frequently to not lose the queue of messages.

\section{Solar Baseline}
\begin{table}[h!]
  \centering
  \begin{tabular}{lcccccc}
  \hline
   & Baseline 0 & Baseline 1 & Baseline 2 & Baseline 3 & Baseline 4 & Agent \\
  \hline
  Transmission Actions & 960 & 480 & 320 & 240 & 190 & 308 \\
  Collection Actions   & 0   & 480 & 640 & 720 & 770 & 652 \\
  Sleep Actions        & 0   & 0   & 0   & 0   & 0   & 0   \\
  Avg. Message Count   & 709 & 766.5 & 769 & 782 & 805 & 790 \\
  Avg. Power Level     & 45  & 52  & 55  & 55  & 57  & 55  \\
  \hline
  \end{tabular}
  \caption{Comparison of Baselines and Agent Performance}
  \label{table:baselines_vs_agent}
  \end{table}
  

% All references should be stored in the file "references.bib".
% That call to use that file is in "cai.cls". 
% Please do not modify anything below this line.
\printbibliography[heading=subbibintoc]

\end{document}
